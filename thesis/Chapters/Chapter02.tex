\chapter{Literature Review}
\label{ch:chap2} 

%------------------------------------------------------------------------------

\begin{chapquote}{Jos\'{e} Saramago, \textit{The Double}.}
Chaos is merely order waiting to be deciphered.
\end{chapquote}

Scholars have long been interested in the culture of prisons \citep[398]{hunt1993changes}. As other subjects in modern sociology and political science \citep[]{hoffmann2000american, munck2008passion}, prison studies started as a mainly American discipline. Therefore, this chapter begins with a brief discussion on the American literature on prison culture and gangs, followed by a review of recent texts on the PCC, largely written in Brazil and so far accessible only in Portuguese. 

\section{Theoretical Models of Prison Behaviour}

The study of inmate culture can be traced back to the 1940s \citep[]{glaser1972bibliography} when \cite{clemmer1940prison} published his pioneer work on inmate life in a maximum security prison in the United States. The author argued that inmate behaviour is a function of the prison environment, and that this structural condition determines inmates' attitudes and mores \citep[62]{craddock1996comparative}. An important concept formulated by \citet[270]{clemmer1940prison} is \textit{prisonisation}, which describes the fact that inmates progressively adopt the subculture of prisons, itself established in order to cope with the difficulties of being incarcerated. \cite{sykes1958society} expands the ``prisionisation theory\footnote{Also known as ``deprivation theory''. See, amongst others, \citet[][]{mccorkle1995roots}.}'' and affirms that, although not as totalising as suggested by \citeauthor{clemmer1940prison}, such process occurs regardless of the type or location of the prison, and that ``this value system takes the form of an explicit inmate code, which is used as a guide for behaviour in inmates' relations with fellow prisoners and guards. Therefore, the inmate code summarises the behavioural expectations of the inmates' social system'' \citep[429]{paterline1999structural}. In a similar vein, other authors also emphasised that the prison is a ``total''\footnote{\citeauthor{goffman1961characteristics}'s work was not particularly concerned with prisons (his book describes the habits of mental patients), but due to the fact that both prisons and asylums were then considered similar examples of total institutions, his ideas were soon used to the analyse everyday practises in jails \citep[]{farrington1992modern}.} \citep{goffman1961characteristics} or a ``complete and austere'' institution \citep{foucault1975surveiller}. \citeauthor{foucault1975surveiller} stressed that Western penal systems had undergone a series of changes over the last centuries, changing its focus from physical punishment of offenders to the use of the prison discipline as a correction measure. In this regard, the author inserts the prison in a network of modern institutions devoted to promote discipline in the Western society (e. g. schools, hospitals, factories, the military), and asserts that the prison system has been used to create a new class of outcasts by severely punishing what was previously considered petty, minor crimes. 

Thus, according to the deprivation theory, the defining characteristic of the prison is that the authorities control all aspects of the inmates' lives tightly, and their subculture is therefore a reflex of such conditions \citep{becker2003politics}.

Recently, scholars have proposed a more nuanced version of this argument \citep{davies1989goffman, thomas1978structural, zamble1988coping}. Critics of the prisionisation theory posit that no convict enters a correction facility as a \textit{tabula rasa}: detainees always bring some of their previous culture and habits into the prison system \citep[109]{huebner2003administrative}. \citet[142--6]{irwin1962thieves}, for instance, suggest that ``[\dots] much of the inmate behaviour classified as part of the prison culture is not peculiar to the prison at all'', and  ``the behaviour of the great majority of men arrested or convicted varies sharply from any ``criminal code'' which might be identified''. \cite{cao1997prison} and \cite{harer1996race} elaborate on this idea and affirm that there is a continuous exchange between prisoners and the outside world, and \citet[655]{delisi2003criminal} notes that many individuals engage and even expand their illegal activities within the jail system. \citet{gambetta2009codes}, in his turn, writes that the penitentiaries are actually a very important tool for criminals to create bonds, assess credible information, and solve coordination problems. Thus, far from being mere reactive agents within a monolithic total institution, prisoners are part of a dynamic environment, formed not only by the structural constrains inmates face, but also by their independent actions \citep{passos2013defesa}.

\section{Research on Prison Gangs}

Although most of the available information about detainees has been generated through prison staff \citep{fong1991detection, gaes2002influence}, there is also a burgeoning literature on prison gangs, the most common form of inmate association. Broadly defined, prison gangs are ``organisations which operate within the prison system as a self perpetuating criminally oriented entity\footnote{There is no universal understanding of what constitutes a crime \citep{gillani2009unemployment, hirschi1990substantive}, and lawyers, economists and sociologists have offered a myriad of definitions to the phenomenon \citep[]{henry2001crime}. In this thesis, I adopt the definition provided by \citet[]{becker1968crime}, which states that a criminal offence is any action that makes the individual run the risk of being condemned to a penalty. \citet[]{foucault2010birth} comments that Becker's definition is not grounded in law, but rather in a rational cost-benefit analysis: punishment is nothing more than a price to be estimated and paid by the (potential) criminal \citep{dilts2009michel, donohue2007economic}. The author also notes that \citet[]{becker1968crime} marks the foundation of the neoliberal approach to crime, where the \textit{homo \oe conomicus} is brought to the centre of criminal studies. Consequently, this definition conforms to the tenets of rational choice theory.}, consisting of a select group of inmates who have established an organised chain of command and are governed by an established code of conduct'' (\citeauthor{lyman1989gangland}, \citeyear{lyman1989gangland}, 89 apud \citeauthor{delisi2004gang}, \citeyear{delisi2004gang}, 371). This definition can be easily extended by asking \textit{under what conditions} and \textit{in what ways} gangs institutionalise their practice and maintain internal order. As described by \citet[1000]{sobel2009youth}, much of the literature on gang creation derives from early works on government formation. Based upon the seminal contributions of \cite{nozick1974anarchy} and \cite{buchanan1975limits}, several authors have highlighted the relationship between anarchy -- at least at the local level -- and the demand for protection as the main driver of gang formation  \citep{bandiera2003land, konrad2012market, skaperdas19973, skaperdas2001political}. The main idea behind this line of thought is that gangs are a sort of ``embryonic state''. \citet[]{sobel2010moregangs} summarises this line of thought: 

\begin{quotation}
That literature suggests that within a society without law and order, individuals are under constant threat of being victims of aggression and crime, and small `gangs' evolve to provide protection services to people. By forming groups, people who cannot protect themselves individually can be more secure; an attack on a single member would result in group retaliation. In other words, individuals form gangs for the same reason that national governments form mutual defence alliances such as NATO.
\end{quotation}

\citet[]{sobel2009youth} point out that the failure of government to protect property rights from violence committed by youth has led to the creation and expansion of youth gangs, a phenomenon which also happens in prisons. Similarly, \citet{skarbek2011governance} writes about how the Mexican Mafia enforces deals and protects property rights in prisons of Los Angeles. The author's main contribution is to analyse how the gang managed to establish a tax structure that enables them to create governance institutions that mitigate market failures and can be mobilised to protect its members. 

Gangs also work as informal ``power brokers'' to facilitate negotiations between two parts. \citet[]{gambetta1996sicilian} and \citet[]{bandiera2003land}, in two classic works about the Sicilian mafia, emphasise that the key business of the mafiosi is not crime \textit{per se}, but protection and mediation. \citet[58]{gond2009reconsidering} write that

\begin{quotation}
In societies with inadequate governance and/or low levels of mutual trust (e.g., underdeveloped or emerging economies characterised by 'weak states'), both parties involved in a market transaction might opt for Mafia protection as guarantee, investing it thus with the role of a profit-making intermediary.
\end{quotation}

\citet[]{varese2001russian} indicates that the Russian mafia operates a similar system of property rights protection. The author charts the emergence of the Russian mafia in the context of the transition to a new market economy, where the abilities of the Russian state to define property rights and protect contracts were still notably weak and there was a strong demand for third-party mediators and law enforcers. 

Drawing upon the literature on social trust and signalling \citep{cook2007cooperation, schelling1980estrategy, schelling2007strategies}, \citet{campana2013cooperation} unpack the role of threats on criminal organisations and describe how violence serves as a credible commitment amongst Italian and Russian mobsters. Although not specifically targeted at prison gangs, their research offers many insights on why criminal organisations use violence. 

\citet{leeson2010criminal} write about an interesting and so far understudied topic: the role of constitutions in gangs. The authors present a case study of \textit{La Nuestra Familia}, a prison gang based in Los Angeles. The authors state that prison gangs use constitutions for three purposes: First, criminal constitutions promote consensus by creating common knowledge about what the gang and other criminals can expect of each other. Second, constitutions are also clearly designed to control individual behaviour and foster collaboration between members. Finally, criminal constitutions generate information about member misconduct and coordinate the enforcement of rules that prohibit such behaviour. In contrast to the general perception on criminal organisations, outlaws have strong incentives to abide by their constitutions since they ``greatly reduce the potential for intra-organisational conflicts'' \citep[282]{leeson2010criminal}.

In a recent working paper, \citet{lessing2014cddrl} shows that common state responses such as crackdowns and harsher sentencing can actually strengthen prison gangs' leverage over outside actors. He argues that the recent increase in encarceration rates in Mexico, Brazil and the United States have given more power to prison gangs, since there is higher demand for protection in the penal system. ``Zero-tolerance'' policies, in this sense, have reduced crime on the streets, but have made prison gangs powerful challengers to the state institutions. 

There are also several case studies on prison gangs. In the United States, there are plenty of texts on the Mexican Mafia \citep{blatchford2008black, maguire1999policing, morrill2005mexican, rafael2013mexican}, La Nuestra Familia \citep{hunt1993changes, koehler2000organizational, lewis1980social}, Mara Salvatrucha \citep{etter2010mara, fogelbach2005mara, grascia2004gang, o2010reckless, wolf2012mara}, and the Aryan Brotherhood \citep{pelz1991right, price2005murder}. Elsewhere, academic research on prison gangs is also growing. There is a sizable literature on gangs from South Africa \citep{dissel2002reform, houston1998prison, lotter1988prison}, Central America \citep{bruneau2011maras, miguel2010central, sullivan2008transnational}, and even from Sweden \citep{larsson2011svensk}. Such studies describe, sometimes with a great amount of detail, how gangs have come to being in their respective countries and point out the inner workings of those criminal organisations. 

Studies on Brazilian gangs deserve special attention in this thesis. Although they are still not numerous, there has been a growing interest in the workings of the two largest Brazilian gangs, Rio de Janeiro's Comando Vermelho (CV) and S\~{a}o Paulo's Primeiro Comando da Capital. Unfortunately, the literature is mostly comprised by texts in Portuguese, what makes it virtually inaccessible to foreign researchers. In the next section, I provide a brief summary of the recent academic contribution to the topic based upon a review included in \citet{dias2011pulverizaccao}. I also comment how the present thesis can help to fill the gaps in prison gang studies in Brazil.

\section{Literature on S\~{a}o Paulo's Prison Gangs}

Regarding the Brazilian academic contribution to the prison gang literature, \citet[365]{dias2011pulverizaccao} notes that although the number of texts analysing empirical aspects of criminal organisations is still small, it is slowly increasing. Not only is there now an important line of works written by investigative journalists on how gangs are formed and maintain internal order \citep{amorim2003cv, barcellos2003abusado, jozino2004cobras, souza2007pcc}, but there is also a growing scholarly literature on drug trafficking in Rio de Janeiro \citep{lessing2008facccoes, zaluar1999debate}, on the rebellions within S\~{a}o Paulo's prison system \citep{adorno2007organized,da2009crime, salla2006rebelioes}, the role of police intelligence in the fight against crime \citep{mingardi2007trabalho}, the relationship between organised crime and the prison system \citep{ramalho1979mundo, pieta1993pavilhao, porto2007crime}, and what matters the most for the present thesis, on the origins and inner workings of the PCC \citep{biondi2010junto, dias2009guerra, dias2011pulverizaccao, marques2010liderancca}.

\citet{thompson1980questao} is probably the first author to deal with ``the penal question'' in Brazil. Highly influenced by the aforementioned idea of \textit{prisionisation} \citep{clemmer1940prison}, \citeauthor{thompson1980questao} indicates that ``the culture of prisons'' is what enables inmates to cope with the hardships of incarceration. However, as such habits are internalised, they make it extremely difficult for a former detainee to go back to life outside of the prison. Moreover, once the prisoner has to adopt the mores of the penal system, it is clear that the prison system does not perform its main task: the re-socialisation of the inmate \citep{silva2011visao}. In this regard, the penal system is in a state of perpetual crisis.

Another pioneering work was written by \citet[]{ramalho1979mundo}. The author conducted extensive fieldwork in S\~{a}o Paulo's largest prison\footnote{The prison was officially called \textit{Casa de Deten\c{c}\~{a}o}, literary ``detention house''. However, it was widely known as Carandiru Penitentiary due to the neighbourhood in which it was located. Carandiru was South America's biggest prison, and at its peak housed over 8,000 inmates. The prison became infamous for being the locus of the killing of 111 inmates in 1992, an event that will be discussed in \autoref{ch:chap3}. Carandiru was demolished in 2002. \citep{ferreira2012massacre, varella1999estaccao}.} and discussed several important aspects of the inmates culture. One innovation brought by \citeauthor[]{ramalho1979mundo} was a spatial analysis of the prison. He noted that there was a sharp dichotomy between ``work'' and ``crime'' in Carandiru. Prisoners who worked in the facility were considered to be apt for social life in a near future, and used to live in pavilions with much better infrastructure. Conversely, hardened criminals, who were seen as ``irredeemable'' by the prison staff, lived in the worst parts of the penitentiary where basic services were rarely provided. However, \citeauthor[]{ramalho1979mundo} writes that, despite its spatial configuration, Carandiru as a whole was remarkably overcrowded, corrupt, dirty and violent \citep[376]{dias2011pulverizaccao}.

\citet[]{coelho1987oficina} completes the triad of the ``classic'' texts on Brazilian prison gangs. Written when the government of Rio de Janeiro tried to open communication channels with the inmates (1980s), the book shows how the ``prisoner commissions'' were soon dominated by gang leaders and used as a tool to repress other detainees. Additionally, the rise of homicide levels in Rio (mainly due to disputes between drug dealers) increased the pressure over the government to adopt a strict approach towards the inmates. 

Likewise, \citet[]{salla2007montoro} describes the evolution of the penal policies in the state of S\~{a}o Paulo from 1982 to 2006. \citeauthor[]{salla2007montoro} notes that although the country was undergoing a process of democratisation, attempts to modernise the prison system were not successful. Whereas two governors tried to adopt a ``more humane'' approach towards inmates in the 1980s, the reforms were rolled back by the hardliners that came to power in the 1990s. The result, according to the author, was the ``Carandiru massacre'' in 1992, where 103 detainees were killed by police forces, the rise of the PCC as a institution to protect prisoners from authority abuse. The state continues to take only small, incremental measures towards the prison population and has not proposed any long term plan to solve the many problems of the penal system \citep[383]{dias2011pulverizaccao}.

\citet[]{alvarez2013comissoes} conducted another study on the changes in the recent penal policies in S\~{a}o Paulo. The authors hypothesise that since the inmates could not formulate legitimate associations to bargain with official authorities, they decided to resort to prison gangs to force the state to the negotiation table. As also mentioned by \citet[]{salla2007montoro}, the conservative governments in S\~{a}o Paulo (1990s) discarded any possibility of legitimate exchange with the detainees. One of the results was the emergence of the PCC as a mediator between the inmates and public authorities. 

\citet{adorno2007organized} affirm that the newly organised networks in Brazil, of which the PCC is the best example, have undergone a process of modernisation in the last decades. The PCC used cell phones, radio and illegal telephone stations to coordinate the rebellions of 2001 and 2006, a strategy soon followed by members of the CV in Rio de Janeiro. The Brazilian state, on the other hand, did not have even official homicide statistics up to 1990, what shows its complete lack of knowledge regarding the criminal world. The increasing power of drug cartels also changed the way criminals operate in Brazil. The drug sale usually demands a high number of criminals and corrupt officials to be profitable, so there was also an organisation impulse to modernise criminal activities in the country.

\citet[]{biondi2010junto} presents a dense ethnography of a PCC-dominated prison, whose access was facilitated by her husband, an inmate. She presents a brief chronology of the PCC and describes the gang's ``ethic'' in detail. She also points out that the main intent behind PCC's actions is to promote equality amongst inmates and guarantee their rights, albeit using violence, against governmental abuse. However, as mentioned by \citet[376]{dias2011pulverizaccao}, by unquestionably accepting the official discourse of the PCC, \citeauthor[]{biondi2010junto} does not pay sufficient attention to the constant use of physical and psychological violence by the cartel. \citeauthor[]{biondi2010junto} does not explain, for instance, why drug addicts, homosexuals and rapists are treated so harshly by the PCC, and even though the \textit{Commando}'s official discourse asks for ``integration and equality'', such prisoners and members of other gangs are frequently punished with severity by the gang. 

The work of \citet{dias2011pulverizaccao} deserves special attention. Her PhD dissertation (later published as a book) describes the origins and the expansion of the PCC. Using data collected after extensive field research in three state prisons, \citeauthor{dias2011pulverizaccao} explains that the PCC exerts control over the incarcerated population mainly through the use of force, and that a new criminal monopoly of force has replaced the Weberian paradigm of state. Whilst the author's theoretical findings are interesting, her principal contribution lies in offering a precise chronology of the PCC, noting the gang's different tactics and goals in its almost 20 years of operations. According to \citeauthor{dias2011pulverizaccao}, the cartel's history has three distinct phases: the ``conquest phase'' (1993--2001), when the gang was centralising their power inside the prison system; the ``public phase'' (2001--2006), when the PCC organised the two rebellions that called the attention of the public and media to its existence; and the ``consolidation phase'' (2006--present), when the gang enjoys undisputed control of the prisons in its native S\~{a}o Paulo state. \citeauthor{dias2011pulverizaccao}' chronology is now widely cited by the specialised literature, and is the one I adopt in this thesis. 

Although the number of academic works on the PCC has increased, we still lack analytical works on the gang. There is not a single text that provides a general, testable framework to analyse PCC tactics and strategies. The present thesis aims to partially fill this gap in the current literature on the PCC. I address two puzzles in the PCC literature: individual incentives to join the PCC and recruitment strategies for the gang. As it is discussed at length on \hyperlink{page.41}{page} Even though the gang has now undisputed dominance over the penal system, only 3\% of the total inmate population is directly affiliated to the cartel. Why not more, if not all convicts? %Secondly, I develop an agent-based computational model to analyse the use of violence by the PCC. Under what conditions would the gang use more lethal violence, ``soft'' punishments or rewards?

However, before moving to the second part, I present a brief history of the PCC, which will also inform the models developed in the last chapters of this thesis. In the \hyperlink{page.19}{next chapter} I describe how the group emerged in the wake of a brutal prison massacre, and how it eventually became Brazil's most powerful inmate gang. As far as I know, no scholar has published a history of the PCC in the English language, so \autoref{ch:chap3} might be also useful as a guide to non-Portuguese speakers.



