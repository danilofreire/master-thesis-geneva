\chapter{Discussion}
\label{ch:chap5}

%----------------------------------------------------------------------------------------

\begin{chapquote}{Thomas Paine, \textit{A Dissertation on the First Principles of Government}.}
He that would make his own liberty secure, must guard even his enemy from oppression; for if he violates this duty, he establishes a precedent that will reach to himself.
\end{chapquote}


The present thesis discussed how the \textit{Primeiro Comando da Capital}, a S\~{a}o Paulo-based prison gang, was able to centralise power and extend its influence to around 90\% of its native state's penal system \citep[]{veja2013}. Recently, the PCC's reach has crossed S\~{a}o Paulo's borders, and it was reported that the cartel has established itself in 22 of Brazil's 27 states, and has even been involved in illegal activities in Bolivia and Paraguay \citep[]{bbc2013pcc}. The PCC has also allegedly been responsible for a remarkable reduction in homicide rates both within and outside the prison system. This is largely in line with theories of state formation. For instance, \citet{buchanan1973defense} argues that a monopoly of violence is better than a free market competition on protection since the monopolist has strong incentives to under-produce violence and allocate those resources elsewhere\footnote{Theoretically, one can say that the underproduction of violence is Pareto superior in a partial equilibrium setting.}. The author writes that after a monopoly has been established ``[\dots] resources involved in enforcement may be freed for the  production of alternative goods and services that are positively valued; the taxpayer has additional funds that he may spend on alternative publicly provided or privately marketed goods and services'' \citep[402]{buchanan1973defense}. Those resources have been mainly invested in the enlargement of the group's stake in the drug selling business in S\~{a}o Paulo, and as we have show in \autoref{fig:fig8}, it seems that such strategy has been largely successful. In this regard, the PCC operates more or less like a ``pseudo-state'' in Brazil's penal system, and seriously undermines the official authority at its very core: the right to punish and protect the social contract \citep[]{hobbes1985leviathan,weber1919politik}.

The rise of the PCC is a good example of unintended consequences of public policies. Whereas the government hoped that its zero-tolerance programme would reduce crime in S\~{a}o Paulo notably violent state, it has fostered the expansion of a much more significant threat to public security, a highly organised and powerful prison gang. Needless to say, the PCC currently represents a much more dangerous threat to public security than the myriad of relatively petty criminals that used to be in charge of S\~{a}o Paulo's illegal markets. In this case, Milton Friedman's famous criticism of public policies seems unfortunately correct. ``One of the great mistakes'', he said in a TV interview with Richard Heffner in 1975, ``is to judge policies and programmes by their intentions rather than their results.''\footnote{The interview can be seen at \href{http://youtu.be/ImMgZHbeb4Q}{http://youtu.be/ImMgZHbeb4Q}. Access: 15th June, 2014.} In this sense, the mass incarceration policy cannot be claimed successful. As Fr\'{e}d\'{e}ric Bastiat insightfully noted more than 150 years ago, every law or institution produces not only effects that are \textit{seen}, but also effects that are \textit{unseen}, which emerge only subsequently \citep[]{bastiat1995selected}. The difference between a good legislator and a bad one is precisely such ability to foresee such unexpected results of public policies; and in S\~{a}o Paulo they were not.

Interestingly, the PCC was able to monopolise power in the penal system having only a fraction of the inmate population in its ranks. The group remains quite small, although its force has increased significantly over the last years. The puzzle I tried to address in this thesis concerns precisely the recruitment process employed by the group. If the PCC offers so many advantages to its members, and it has clear expansionary intents, why is not every detainee part of the gang? Those questions, as obvious as they may seem, had never been addressed by the specialised literature.  Therefore, albeit very modestly, this thesis collaborates to our current understanding of prison gangs and their selection criteria, and also sheds some light on what are the individual incentives to join such criminal groups.

The model has three important findings. First, it shows that a prison gang uses the value of its initial costs as a first selection method, since only skillful prisoners will be able to comply with the gang's demands if the costs are too high. Although useful, this procedure generates costs for both the prisoner and the gang. For the prisoner, it is necessary to spend long periods of time to display all abilities required by the gang, what may lead to an increasing lack of interest for the criminal. The gang also has to bear costs, since the process of establishing and enforcing a high threshold demands considerable human and financial resources from the group. In order to reduce such gang costs, the group can make use of informers. With an informer, the gang can keep a lower entry cost, thus attracting a large pool of applicants while still being able to select competent candidates for the job. 

Second, there are cases in which joining a prison gang is not the best option for an inmate. When the detainee has enough skills to endure prison conditions by himself, he might be better off if he decides to ``go it alone'' and devote his ability exclusively to his own survival. If a prison gang decides to lower its threshold to admit a higher number of members, a competent criminal will have its individual benefit reduced. Since the collective goods provided by the gang would be divided amongst several detainees, his net benefit will be lower than in the previous stage, where the selection process was stricter. However, the gang can eliminate this loss by raising the bar for admission, so that qualified criminals will only share the collective goods with those who can contribute to the group's welfare as much as they do. At its early stages, the PCC indeed followed such strategy in order to maintain the group cohesive enough to carry risky actions without losing highly-skilled members. After the group had already secured its dominance, it was then able to open its ranks to less skilled members and proceed with its model of ``careers open to talent'' (see page \hyperlink{page.33}{33}).

Third, the models confirms the idea that the prison gang is not only a ``school of crime'', but perhaps most importantly, a highly effective screening device \citep[]{gambetta2009codes}. If a given prisoner knows that the admission to a certain gang is difficult to obtain, as soon as he manages to join the gang he can identify peers who are at least as skillful (and trustworthy) as he is. Therefore, prison gang membership solves one of the most critical issues of the criminal underworld, that of \textit{peer identification}. If a prison gang's admission process is deemed as satisfactory by the prisoners, if an inmate joins the gang he sends a signal that only true criminals could afford, thus dramatically reducing the chances that he might be a snitch. This also has the benefit of reducing violence between criminals, as they do not have to constantly ``test'' each others' integrity or to use violence as a communicative means to express one's toughness. Prison gangs thereby increase the welfare of the inmates by providing an extremely valuable public good: reliable information.

Two policy implications can be derived from the findings discussed above. One important suggestion is that states should reduce the total amount of benefits that prison gangs can offer to its members. The most evident manner by which this can be done is by improving prison conditions. For instance, reducing overcrowding in prisons would decrease violence between inmates by decreasing the likelihood of violence disputes for scarce resources such as food, health care and individual space within the cells, all things that are currently being distributed by the PCC in S\~{a}o Paulo \citep[]{dias2009ocupando}. Moreover, if the state is able to reduce police abuse it will lower the need for a crucial public good provided by prison gangs: protection. The more states care about prisons conditions, the less necessary prison gangs are.

Also, the present dissertation questions the recent mass incarceration policy implemented in the state of S\~{a}o Paulo. Such policy has caused devastating effects to the poor population, who has repeatedly been the main victim of both institutional and criminal violence. In this sense, this thesis shows the urge for a discussion on the hidden but persistent channels by which inequality has been perpetuated in the Brazil. 

Finally, it should also be noted that this study can be expanded in several ways. The model shown here was based upon information gathered on the workings of the PCC, and it has to be tested with data from other prison gangs in order to give it external validity. Moreover, due to serious lack of data on the topic, I was not able to incorporate the state dimension into the model. By including corrupt prison officers and weak public institutions to our formal analysis one could make the model even more realistic and derive more robust predictions from subsequent findings. It would lead to a better-informed discussion on the efficiency of the public institutions in Brazil's recent history, and maybe shed some light on the dark side of country's current democratic regime.















%In mathematics, physics and biology, computational models have long been an important method in the researchers' toolbox \citep[71]{box2008oxford}. The increase in computing power over the last decades has allowed for otherwise intractable problems to be solved via simulation. For instance, the advent of Markov Chain Monte Carlo techniques -- implemented by Gibbs sampling or the Metropolis-Hastings algorithm \citep[]{gelman2013bayesian} -- has helped to solve multi-dimensional integrals present in Bayesian analysis, and they were largely responsible for the growing adoption of Bayesian statistics in a vast number of academic fields \citep[]{mcgrayne2011theory}. 

%Agent-based models (ABMs), in this sense, present a new set of possibilities to researchers in general and to political scientists in particular \citep[]{kollman2006computational}. Agent-based models are composed by a set of autonomous, interacting computational objects, known as agents, whose behaviour is rule based \citep[2]{de2014agent}. Agents can be individuals, firms, governments, institutions, or any other category researchers want to analyse. Networks, identities and learning mechanisms can also be embedded in ABMs, thus making the models apt for testing non-linear, path-dependency dynamics in complex systems \citep[3]{bhavnani2004agent}. ABMs have been seen as a ``revolutionary tool'' for the social sciences \citep[]{Bankes14052002}: several social phenomena can be modeled as non-equilibrium, dynamic settings, and scholars can simulate social processes and carry out "experiments" that would otherwise be impossible \citep[]{davidsson2002agent}. 

%\Citet[2--6]{de2014agent} point out that ABMs have six important capabilities: 1) ABMs allow researchers to link behavioural rules to aggregate patterns, be they equilibria or complex patterns; 2) ABMs enable the inclusion of geographic and social space at multiple levels of resolution; 3) ABMs have the capacity to include multiple processes simultaneously; 4) ABMs can include heterogeneity not only in location, beliefs, information, and ability, but also in learning rules, perspectives, mental models, behavioural repertoires , and cognitive framing; 5) ABMs prove capable of producing messy contingent outcomes and a range of phenomena: randomness, equilibria both static and distributional, patterns and complexity, and the complex outcomes can vary in their amount and type of path dependency; and 6) analysis of ABMs tends to focus on outcome robustness, i.e., whether the system can keep doing what it needs to do as opposed to the existence and efficiency of equilibria. To this list we should add two other advantages of ABMs that are particularly relevant for the present research. ABMs can present results in a clear, simple way. Simulation graphs convey the same information contained in game-theoretical mathematical proofs, but have a much more intuitive interpretation. Also, AMBs give us the opportunity to test mechanisms that are difficult to observe due to the inherently secretive and dangerous nature of organised crime. ABMs are therefore the research method of choice for evaluating our hypotheses.

%To the best of my knowledge, there is not a single paper discussing the PCC which employs formal or computational methods. Thus far, no generalisable set of propositions using the PCC as an example has ever been formulated, and this thesis aims to fill this gap. In the following sections I present the a model to analyse individual choices in a prison environment, and evaluate some comparative statics of the model's parameters.