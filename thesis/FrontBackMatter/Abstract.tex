% Abstract

\pdfbookmark[1]{Abstract}{Abstract} % Bookmark name visible in a PDF viewer

\begingroup
\let\clearpage\relax
\let\cleardoublepage\relax
\let\cleardoublepage\relax

\chapter*{Abstract} % Abstract name

The present thesis provides a throughout discussion of the emergence of the \textit{Primeiro Comando da Capital} (PCC), a prison gang based in S\~{a}o Paulo, Brazil. Its main goal is to analyse how this criminal group selects its potential members. The work starts with a review of the recent literature on prison culture and gangs, with special emphasis on the Brazilian contributions to the field. Then it presents the first historical account of the PCC in the English language since previous research has been solely conducted in Portuguese. Lastly, the thesis offers a simple game-theoretical model to analyse both the incentives for a criminal to join a prison gang and how the PCC has been able to hire competent criminals under conditions of uncertainty and information asymmetry. The model suggest three findings. First, it stresses the role of informers in the gang's recruitment process. Informers allow the prison gang to keep a lower entry cost, so the gang can attract a larger pool of applicants and still be able to select competent candidates. Second, it indicates that 
there are cases in which joining a prison gang is not the best option for an inmate. When the detainee has enough skills to endure prison conditions by himself, the prisoner might be better off if he decides to ``go it alone'' and devote his ability exclusively to his own survival. Third, the models confirms the idea that the prison gang is not only a ``school of crime'', but perhaps most importantly, a highly effective screening device. Prison gangs increase the welfare of the inmates by providing an extremely valuable public good: reliable information. Public policies implications and possible extensions of the current study are also discussed.\\



%This criminal group, which currently extends its influence to around 90\% of its native state's penal system, was able to monopolise power in prisons by having only 3\% of the inmate population in its ranks. In the present work I develop a formal model to address this question and derive a few conclusion from it.

\noindent
\textsc{Keywords:} Organised Crime, Prison Gangs, PCC, S\~{a}o Paulo, Game Theory

\endgroup			

\vfill