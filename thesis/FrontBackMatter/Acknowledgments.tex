% Acknowledgements

\pdfbookmark[1]{Acknowledgements}{Acknowledgements} % Bookmark name visible in a PDF viewer

%\singlespacing
%{\slshape  
%All human behaviour is scheduled and programmed through rationality. There is a logic of institutions and in behaviour and in political relations. In even the most violent ones there is a rationality. What is most dangerous in violence is its rationality. Of course violence itself is terrible. But the deepest root of violence and its permanence come out of the form of the rationality we use. The idea had been that if we live in the world of reason, we can get rid of violence. This is quite wrong. Between violence and rationality there is no incompatibility.} \\ \medskip
%--- Michel Foucault, \textit{Truth is in the Future}, p. 299.


\bigskip

%----------------------------------------------------------------------------------------

\begingroup

\let\clearpage\relax
\let\cleardoublepage\relax
\let\cleardoublepage\relax

\chapter*{Acknowledgements} % Acknowledgements section text
There are a number of people to whom I have to express my most sincere appreciation. First and foremost, I would like to convey my gratitude to my supervisor, Professor Ravi Bhavnani, for his guidance and support over the last two years. His contributions are felt throughout the text. I also thank Professor C\'{e}dric Dupont for being the second reader of this thesis.

Several people generously provided comments on this work. I owe thanks to Guilherme Arbache, Fabio Barros, Gustavo Burgos, Guilherme Duarte, Manoel Galdino, Le\^{o}ncio Junior, Sam Keb, Eben Kuni, Davi Moreira, Rafael Nunes, Pietro Rodrigues, Luis Felipe Serrao, Damien Someil, and Sabine Waltraut for their valuable contributions. My friend Umberto Mignozzetti deserves a special mention. Not only he helped me to translate my ideas into proper mathematics, but also made many insightful suggestions to  this dissertation. I am glad to have written my first academic papers with such a brilliant co-author, and I hope we continue to work together for many years. Of course, all faults remain my own.

I have benefited immensely from the help of many members of the open source software communities I am proud to be part of. Thanks to all those who anonymously answered my questions about computer programming, \LaTeX, \texttt{R}, and statistics on R-help and Stack Overflow. 

I also thank to my flatmates Kamel Belmkadden, Raquel Ermida, Jo\~{a}o Matos and Carlos Montesinos for the great time we had together in Geneva.

Guilherme Kerr, Gustavo Kerr and Mauricio Pantale\~{a}o, old brothers in arms, have always stood by my side in good and bad moments. This time it was no different. Thank you very much for the camaraderie. 

No acknowledgement would be complete without giving thanks to my dear Mira. She spent countless nights helping me to clarify my ideas and was always my support when there was no one to answer my queries. Vielen herzlichen Dank, Mimi.

Last but not least, my family has given me all support I needed to pursuit this project. My deepest thanks to my mother Rosali and my grandmother Maria for their unending trust, patience and love. De cora\c{c}\~{a}o, muito obrigado por tudo. Amo voc\^{e}s.



%Put your acknowledgements here.\\

%\noindent Many thanks to everybody who already sent me a postcard!\\

%\noindent Regarding the typography and other help, many thanks go to Marco Kuhlmann, Philipp Lehman, Lothar Schlesier, Jim Young, Lorenzo Pantieri and Enrico Gregorio\footnote{Members of GuIT (Gruppo Italiano Utilizzatori di \TeX\ e \LaTeX )}, J\"org Sommer, Joachim K\"ostler, Daniel Gottschlag, Denis Aydin, Paride Legovini, Steffen Prochnow, Nicolas Repp, Hinrich Harms, Roland Winkler,  and the whole \LaTeX-community for support, ideas and some great software.

%\bigskip

%\noindent\emph{Regarding \mLyX}: The \mLyX\ port was intially done by
%\emph{Nicholas Mariette} in March 2009 and continued by
%\emph{Ivo Pletikosi\'c} in 2011. Thank you very much for your work and the contributions to the original style.

\endgroup