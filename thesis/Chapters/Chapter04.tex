\chapter{Modeling Prison Gang Recruitment}
\label{ch:chap4}

%----------------------------------------------------------------------------------------

\begin{chapquote}{Diego Gambetta, \textit{Codes of the Underworld}.}
Criminals embody \textnormal{homo \oe conomicus} at his rawest, and they know it.
\end{chapquote}

Criminal games are played for very high stakes \citep[2]{dixit2011game}. In the underworld, rules tend to be enforced by the use of lethal violence, therefore mistakes are rarely tolerated and commitments have notably high exit costs \citep{campana2013cooperation}. Prisons take those situations to the extreme \citep{sykes1958society}. Since prisoners share the same space for long periods of time, their interactions are maximised and the chances of fleeing and escaping retaliation are quite low. Moreover, prisoners have their schedules tightly controlled by correctional officers, what imposes clear limits on the detainees' ability to meet and negotiate \citep[7]{skarbek2014social}. If communication between inmates is limited, they usually have to rely upon costly signals such as acts of brutality or body marks, which are by definition inconvenient for the prisoners \citep[]{gambetta2009codes}.

It is also obvious that prisoners often cannot ask for penal institutions to solve their issues, even if they wanted to. Firstly, the penal system has limited resources, so surveillance is imperfect. There is a very small guarantee that the guards will be able to protect the inmates as well as they should \citep[20]{skarbek2014social}. In places like S\~{a}o Paulo, where prisons are overcrowded and officers are knowingly corrupt \citep[]{darke2013inmate, lemgruber2005brazilian, silveira2007realidade}, this is indeed the normal state of affairs. Secondly, even if the penal authorities were honest, officers may not have information on the illegal activities carried by prisoners, such as drug selling or requests for target killings \citep[]{kauffman1988prison}. Thus, they would be unable to mediate or prevent this sort of conflict. However, being seen as an informer in the prison seriously increases the chances that the inmate will suffer severe retaliations \citep[]{aakerstrom1986outcasts}, so even if the guards would have this kind of information it would be unlikely that inmates would ask them for help. Therefore, in the criminal world one is can usually can count only on oneself, and the choices a prisoner makes have serious consequences. What would then be the best strategy for an inmate to adopt?

As we have seen in this thesis, prison gangs are to a large extent a response to these problems \citep[]{camp1985prison, fleisher2001overview}. Gangs reduce transaction costs, enforce physical and property rights protection, and behave as brokers in correction facilities \citep{buentello1991prison, delisi2004gang, skarbek2011governance}. Nevertheless, prison organisations also have their own interests, and they need to maximise their strength by recruiting prisoners who are the ``most competent'' in the underworld. This is definetely not an easy task. 

In the following pages I present a simple formal model to address this question. My intention here is to devise an analytical framework that might explain a criminal's choice to join a prison gang, and how the organisation reacts to that. I have based my game-theoretical model here upon qualitative evidence from the PCC, either gathered by newspapers or scholars, and tried to stick as closely as possible to the group's own categories and actions.

\section{The Model}

The model consists of two players, a prisoner $P$ and a prison gang leadership $G$\footnote{I use a notation similar to that employed by \cite{lessing2014cddrl}.}. Since the purpose of the present exercise is to evaluate the likelihood of $P$ joining a gang, I shall assume that $G$ is already present in a given prison. Therefore, $G$ moves first, and sets a cost $\tau$ to be paid by the prisoner. This a very common PCC practice: not only the need for financial contributions are specified in the article number 7 of their statute (see page \hyperlink{page.27}{27}), but it has been reported by S\~{a}o Paulo's most important newspaper that PCC members are now asked to pay 50 Reais  (about US\$ 25) per month when in jail, and 500 Reais when free \citep[]{folha2006criada}. Also, PCC members are expect to engage in the group's risky plans, so other important non-monetary costs are also captured by $\tau$. Here I assume that such cost is known by $P$, either because the information is described in the PCC statute (see page \hyperlink{page.24}{24}), or because it has been disseminated by the inmates in a PCC-controlled prison.

The prisoner can either comply ($C$) and bear the costs $\tau$, or defect and be independent in the prison ($D$). $G$ plays next, and after assessing if $P$ meets its criteria, chooses whether to accept ($A$) or reject ($R$) the new member, and payoffs are realised.

\begin{figure}[htp]
\centering
\begin{tikzpicture}	[solid node/.style={fill,circle,inner sep=1.5pt}, hollow node/.style={fill,circle,inner sep=1pt}, scale=1.5,font=\footnotesize]
\tikzstyle{level 1}=[level distance=15mm,sibling distance=40mm]
\tikzstyle{level 2}=[level distance=15mm,sibling distance=30mm]
\tikzstyle{level 3}=[level distance=15mm,sibling distance=20mm]
\node (root)[solid node, label=above:{$G$}] {}  
    child {node (a1) [solid node, label=right:{$P$}] {} 
    child{node[hollow node, label=below:{$\left(0,0\right)$}]{} edge from parent node[left]{$D$}}
   child{node(3)[solid node, label=right:{$G$}]{} 
    child{node(4)[hollow node, label=below:{$\left(0,0\right)$}]{} edge from parent node[left]{$R$}}
    child{node(5)[hollow node, label=below:{$\left(1,1\right)$}]{} edge from parent node[right]{$A$}}
    edge from parent node[right]{$C$}
	}
};
\node [draw=none, shift={(.3cm,-1cm)}] (root) {$\tau$};
	\end{tikzpicture}
\caption{Gang Prisoner Acceptance}
\end{figure}

Let $S$ be a measure of the gang's capacity to ameliorate prison conditions. I assume that the true value of $S$ is known to both the prisoner and the gang, what implies that the players have a perfect understanding of how $G$ can help the $P$ by providing both public and club goods. The assumption is reasonable. The benefits given by the PCC, such as protection and financial assistance, are widely advertised in prisons and they are precisely the reason why inmates would join the group \citep{folha2012pendrive}.

Let $s \sim U \left[0,1\right]$ be a series of skills that are useful for prison life. These can be physical strength or other abilities such as strategic thinking, technical expertise or a dense personal network. The prisoner knows his own value $s_P$, whereas $G$ only knows that that $s$ follows a uniform distribution. The utility of $s_P$ increases monotonically, as more skills are always preferred to less. 

The expected net utility can thus be represented as a simple difference between the benefit ($b$) of being in a prison gang and the benefit of remaining independent. As we have seen above, the costs for a prisoner to be in a gang are higher are also positive, since the prisoner has to pay a monthly monetary tax to the PCC and also follow the group's rules and take part in their actions. 

\begin{align}
b_P\left(s,1\right) - b_P\left(s,0\right) &\geq 0\\
c_P\left(s,1\right) - c_P\left(s,0\right) &\geq 0
\end{align}

Nevertheless, there exists a value $\overline{s}$ such that

\begin{align}
b_P\left(s,1\right) - c_P\left(s,1\right) &\leq b_P\left(s,0\right) - c_P\left(s,0\right)
\end{align}

If $s_P \geq \overline{s}$. That is, above a given value of $s$ it is interesting for a prisoner to ``go it alone.''\footnote{Obviously, the opposite is true whether $s_P < \overline{s}$.} If $P$ has enough skills to survive in prison by himself, $P$ might ponder if it is really profitable for him to join any kind of group. For prisoners who are below this threshold, becoming a member is a dominant position. The benefits are therefore $S_P$ = $1$ if the prisoner decides to comply with the gang's cost $\tau$ and the group accepts him as new member ($A_P$ = $C$ and $A_G$ = $A$). In contrast, $S_P$ = $0$ when $A_P$ = $D$ or $A_G$ = $R$, that is, when the prisoner defects or the gang refuses to have in its ranks. 

As for the gang, there is a low threshold $\underline{s}$ under which any member acceptance implies more costs than benefits to the gang. Any prisoner below this value is likely to become a burden to $G$ since he will not be competent enough to fulfil the tasks demanded by the group, and might lower the collective welfare by increasing other members' chances of being punished by the police. Therefore, there exists $\underline{s} \in \left[0, 1\right]$ such that

\begin{align}
b_G\left(s,1\right) - c_G\left(s,1\right) &\leq b_G\left(s,0\right) - c_G\left(s,0\right)
\end{align}

When $s\leq \underline{s}$, and reverse situation when $s \geq \underline{s}$. $G$ will only choose a member that maximises its utility (a high value of $s$), and reject every prisoner it sees as unfit for the job due to his lack of competence in criminal activities.

\subsection{Perfect Information Game}

Assuming that the prisoner and the game have access to complete, there are three possible equilibria to this game. Whether $s_P \in \left[0, \underline{s} \right]$, $P$ will comply and try join the group ($C$), while $G$ will refuse him ($R$). If $s_P \in \left[\underline{s}, \overline{s} \right]$, the equilibrium is ($C$, $A$), in which $P$ joins the group and $G$ accepts him. Finally, if $s_P \in \left[\overline{s}, 1 \right]$ the equilibrium is ($D$, $A$), and $P$ will remain alone. 

This situation would be valid if $G$ could easily assess the qualities of every potential member. While prison gangs do their best to obtain credible information on the convicts, it is likely that a perfect information game would only be a good approximation to the facts in a very selected number of cases. It is true that prison records provide a reliable source of information on individuals, and gangs often have access to those data via inmates and sometimes even prison staff members \citep[]{gambetta2009codes}. Not only this is a long, costly task, but may not be feasible in some circumstances. High security prisons makes this comprehensive evaluation process virtually impossible. In such environment, the gang cannot be completely sure of the inmate's past behaviour and present intentions, and since interactions between a potential gang member and their ``godfathers in crime'' (see page \hyperlink{page.38}{38}) are relatively few, criminals have difficulties to advertise their credentials to the gang. Recruitment with perfect information is therefore unlikely to happen.

In this regard, I now discuss the recruitment process of a prison gang under conditions of information asymmetry. This is a more realistic assumption, and PCC's history confirms its plausibility. The gang's requirement that every prospective member should receive the approval of two ``brothers'' (current members) before joining the gang is a proof that the gang needs to cross-check information before taking a decision. Thus, the \textit{informants} play a crucial role in PCC's recruitment process, what is indeed expected since an inmate's commitment to the gang is for life. Also, informants have strong incentives to provide reliable information to the PCC, simply because they are held accountable for future actions of their ``godson''. Bad selection choices can not only cost him prestige, but to the extreme lead to severe punishments as fines and expulsion from the group. 

\subsection{Imperfect Information Game}

The next game has the same two players of the previous one, a prisoner $P$ and a prison gang $G$. However, I add a third figure that does not take any action in the game \textit{per se}, but is of extreme importance to the model: the \textit{informer} ($I$). Here, $I$ represents the current PCC member who is in charge of assessing the criminal qualities of a potential candidate. While $I$ has several reasons to provide credible information to the PCC, he can also incur in errors and make wrong judgements about a prisoner's abilities. The information about the prisoner can thus be ($T$) or false ($F$)\footnote{By false I do not imply that wrong information given by $I$ is intentionally manipulated. It can be merely the result of miscalculations, and the term here has no pejorative meaning.}.

To reiterate, the game goes as follows.

\begin{enumerate}
\item Nature chooses the skills of the prisoner $P$, with $s \sim U[0,1]$.
\item The gang leader chooses $t^*$.
\item The prisoner decides if apply or not to the gang. If not, the game ends. If yes, the game follows to the next stage.
\item The gang leader then decides to accept or reject the join proposal. The game ends.
\end{enumerate}

In the present model, I assume that the gang will decide whether it accepts the candidate or not based upon his additional productivity. Therefore, the gang will add a new member if and only if

\begin{align}
\alpha \left( n + 1 \right) \geq \alpha \left(n\right)
\end{align}

Where $\alpha$ is the productivity function and $n$ is the current number of members in the gang. We assume thereafter that exists a number of prisoners that optimises $\alpha(n)$, and we denote it as $\alpha^*$.\footnote{Notice further that the productivity can be expressed by a simple function that has limit $1$ when $n \rightarrow \infty$ and limit $0$ when $n \rightarrow 0$. We assume that this function has a maxima in $n^*$ as well. An example could be $(an)^\frac{b}{n}$, for parameters $a$ and $b$ given.}

The game can be solved via backward induction. The prisoner applies when

\begin{align}
U_P [Comply] \geq U_P [Defect]
\end{align}

The utility of joining is equal to

\begin{align}
\alpha (n+1) s - \tau
\end{align}

Where $\alpha (n+1)$ is the marginal productivity of the new member to the gang, $s$ is the skill of the prisoner and $\tau$ is the fee (monetary and non-monetary) imposed by $G$ on $P$. The prisoner's utility for not joining the gang is simply $s$, which is $P$'s individual ability to survive in prison.

We can then expand the equations as

\begin{equation}
\begin{split}
\alpha (n+1) s - \tau &\geq s\\
[\alpha (n+1) -1] s &\geq \tau\\
s \geq \frac{\tau}{\alpha (n+1) -1} & = s^*(\tau)
\end{split}
\end{equation}

All prisoners with $s \geq s^*(\tau)$ will try to enter the gang. Notice that

\begin{align}
\frac{ds^*(\tau)}{d\tau} = \frac{1}{\alpha (n+1) -1} > 0
\end{align}

Thus, the proportion of prisoners that intend to join the gang shrinks as the leadership rises the fee. In such scenario, $G$ can choose qualified candidates only by increasing the entry barriers. 

Let us now consider that $\tau = \underline{\tau} + t$, where $\underline{\tau}$ is an exogenous cost of joining the gang. To illustrate this situation, imagine that police forces might initiate a crackdown on the prison gang, or that there is a dispute between two different gangs in the same jail. In such scenario, $t$ is the share of the cost that $G$ is imposes in the selection process. The idea is to raise the level of its members, since more skills are needed. $G$ will choose a value of $\tau$ that maximises its revenues and skills. This is given by

\begin{align}
\alpha (n(s^*(\underline{\tau} + t))) \times (1-s^*(\underline{\tau} +t)) \times E(s^*(\underline{\tau} + t))
\end{align}

In which the first part of the equation is the effect caused by the productivity parameter, the second part is the proportion of prisoners that are accepted in the gang and the third the expected productivity of the gang as a whole. Furthermore, as the number of members is always decided by criteria of optimise the productivity $\alpha(n)$, we can without loss of generality change it by $\alpha^*$. Thus, the gang optimises the ex-post utility by maximise the following equation

\begin{equation}
\begin{split}
\max_{t \geq 0} \{ \alpha^* \times (1-s^*(\underline{\tau} +t)) \times E(s^*(\underline{\tau} + t)) \} \\
s^*(t) = \frac{1}{3}
\end{split}
\end{equation}

And solving for $t$ we have that

\begin{equation}
\begin{split}
t^* = \frac{1}{3}(\alpha^*-1) - \underline{\tau}
\end{split}
\end{equation}

And therefore, the duple $t^*$ and $s^*(t^*)$ solves the game for the Gang and the Prisoner, respectively.

This equilibrium has interesting qualitative properties:\\

\noindent 
1) All prisoners with $$ s \geq s^*  = \frac{\underline{\tau} + t^* }{\alpha (n + 1) - 1}$$

Will join the prison gang $G$.\\

\noindent 
2) $t^* = 0$, when the exogenous shocks are high\\

\noindent
3) $t^* > 0$ when there are many prisoners trying to join $G$, or then $\underline{\tau}$ is low. When a high number of prisoners are willing to join the prison gang, it can be implied that low-skilled inmates will also apply and get the spot.\\

\noindent
4) $n^*$ maximises the ex-post benefit for the gang. The reason is that, given that $t*$ had already been chosen, the only preoccupation for the gang is its productivity increase. As long as the new member $P$ can rise $G$'s productivity levels, the gang will accept him.\\

\noindent
5) $t^*$ maximises the ex-ante utility for the gang. The reasoning behind it is that the gang can control the admission of prisoners by selecting only inmates with high values of $s*$.\\

Now, consider the situation when we add the informant. The informant acts as a probability of find out the true value of $s$ of a given applicant. Let this probability be denoted by $\pi$, when the complement means that the information is wrong.

The time-line for the second game follows below.

\begin{enumerate}
\item Nature chooses the skills of the prisoner $P$, with $s \sim U[0,1]$.
\item The gang leader chooses $t^*$.
\item The prisoner decides if apply or not to the gang. If not, the game ends. If yes, the game follows to the next stage.
\item With probability $\pi$ the gang leader finds out the true type of the prisoner. With probability $1-\pi$ the gang leader does not find it.
\item Finally, the gang leader then decides to accept or reject the join proposal.
\end{enumerate}

Adding this new feature the the previous game empowers the gang leader by giving him a better screen process, where he can combine fees and a selection upon the revealing of the quality parameter by the informant.

Solving the game backwards we have that the gang accepts an offer, ex-post the choice of $t$ whenever

\begin{align}
\alpha (n+1)S + \alpha (n+1) \left[\pi s + (1-\pi) E[s*|t] \right] \geq \alpha (n)S
\end{align}

Where $S$ is the sum of the abilities of all current gang members. Solving this equation we have that the gang accepts ex-post the new member when

\begin{align}
s \geq \frac{S}{\pi} \left[\dfrac{\alpha(n)-\alpha(n+1)}{\alpha(n+1)}\right] - \dfrac{1-\pi}{\pi}\left[\dfrac{1-s^*(t)^2}{2}\right] = s^{**}
\end{align}

This equation means that there will be a new limit, $s^{**}$ that, upon the information of the productivity of the prisoner, the gang decides by imposing a higher threshold. This means that $s^{**} > s^*(t)$. The qualitative difference that this will generate is that the gang now can select more skilled prisoners without having to raise the fees $t$. 

In the sequence, the prisoner ask to join the gang if and only if

\begin{equation}
\begin{split}
U_P[Comply] &\geq U_P[Defect]\\
\alpha (n + 1) s - \tau &\geq s \\
s \geq \frac{\underline{\tau}+t}{\alpha (n+1) - 1} &= s^*(t)
\end{split}
\end{equation}

Which is the same as in the previous model. Finally, the ex-ante optimum for the gang will be to set the fees at zero. This because when there is informants inside the process we have that the ex-ante value that the gang maximizes is equals to 

\begin{equation}
\begin{split}
\max_{t \geq 0} \{ \alpha^* \times (\pi(1-s^{**})\int_{s^{**}}^1 \sigma d\sigma + (1-\pi)(s^{**}-s^{*})\int_{s^{*}}^{s^{**}} \sigma d\sigma ) \}
\end{split}
\end{equation}

And optimising in $t$ this expression leads us to 

\begin{align}
t = \left(-\frac{s^{**}}{3} - \underline{\tau} \right) (\alpha^*-1)
\end{align}

And as $t<0$, $t^*=0$ for all parameters, as long as $s^{**}>s^*$.

For the model above, we have the following qualitative properties:

\noindent
1) Information minimises the use of entry barriers by the prison gang;\\

\noindent
2) The quality of the information ($\pi$) provider by the informer $I$ also matters. We see that $\pi$ has a positive impact on $s**$, as $\frac{ds**}{d\pi} > 0$. In a nutshell, it is easy to observe that the more (and better) information available, the more precise the screening process is;\\

\noindent
3) In the present model, the proportion of prisoners remains the same;\\

\noindent
4) However, the difference we observe here is that the gang has now a much more qualified cadre. If $G$ is capable of extracting good, reliable information, it has a significant impact on the abilities of the selected members.

\subsection{Comparison between Models}

The first and more important different between models 1 and 2 is that in the latter the selection process is notably more sophisticated than in the earlier one. The use of information, here designed by $\pi$, helps the gang to choose more qualified members to its ranks.

When we reach the limit satisfaction, that is when $\pi \rightarrow 1$ (when the gang's information of the given prisoner is precise and sufficient), the revenue for the gang is clearly maximised. In a nutshell, for a prison gang information is a very necessary good, certainly crucial for it to recruit the correct prisoner\footnote{The true role of information may be distorted in these conclusions, but note that $\pi$ also changes the second threshold $s^{**}$. This makes information the key to the last result.}.

In the first model, one may observe that $t^* \geq 0$, whereas in the second one $t^*=0$. This is a result of information, which is enough to deter unskilled prisoners from entering the gang.

Interestingly, not only $G$ has a benefit increase when information is added to the model. For $P$, adding information to the selection process eliminates the cost $t^*$, thus reducing the loss for the inmate.

Another interesting point is that such result holds for general $\alpha^*$ values as long as we consider that $\alpha(n)$ has those characteristics assumed in the text. If we assume that $\alpha$ reaches a point where it starts to generate benefits lower than $1$, some prisoners will prefer to ``go it alone'' and not join the prison group. Obviously, the most skilled prisoners will be these ones that shall run alone and the others may free ride, as they are less skilled and can free ride in the group's productivity.

Finally, if we consider the possibility of a lower threshold $\underline{s}$ under which the gang prefers to refuse a prospective member, if $s^* < \underline{s}$, there is the possibility that the gang is indifferent for that choice. In the informer's equilibrium, there is a $\pi$ probability that the inmates will get caught by a third party (the police forces or a rival gang in our example), and shall not join the gang. This measure avoids the fact that the gang leadership would become weaker.



%%%%%%%%%%%%%%%%%% The following paragraph is now useless, but it might
%I also present a slight change to the model. Instead of having a continuous distribution of $s_P$ (a set of $P$'s criminal skills), in this part I have three discrete categories in this parameter, so that $s$ = $\{0,1,2\}$. This classification also mirrors the manner by which the PCC refers to prisoners in general, as a ``thing'' (\textit{coisa}, former member of a rival gang or untrustworthy criminals), a ``thief'' (\textit{ladr\~{a}o}, a competent criminal), or a ``rascal'' (\textit{malandro}, criminals who are either very smart or want to take advantage of other inmates.) Such categorisation maintains the same idea presented before, that there exists a value where $s$ is too low for $G$ to accept $P$ as a member, and that there is high threshold of $s$ which creates incentives for $P$ to not join the prison gang. However, the classification into categories makes the model easier to understand, since it simplifies the calculations without loss of generality.
